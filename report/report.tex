\documentclass[11pt]{article}
\usepackage{amssymb}
\usepackage{algpseudocode}
\usepackage{algorithm}
\usepackage{setspace}
\usepackage{graphicx}
\graphicspath{ {./images/} }
\usepackage{hyperref}
\usepackage{siunitx}
\usepackage{amsmath}
\usepackage{caption}
\usepackage{subcaption}

\hypersetup{
    colorlinks=true,
    linkcolor=blue,
    filecolor=magenta,
    urlcolor=cyan,
}

\title {Ranking Loss Surrogates \\[1ex] \large MSc Thesis}
% Commenting out another way to write the title.
% \title{MSc Thesis - Ranking Loss Surrogates}

\author{
        Abdus Salam Khazi [
        \href{mailto:abdus.khazi@students.uni-freiburg.de}
                {Email} ]\\ \\
        \href{https://github.com/abduskhazi/ranking-loss-surrogates.git}
                {Github Repository} \cite{github_repository} \\ \\
        Supervisors:
        \begin{tabular}{ll}
             JProf. Josif Grabocka \&
			Sebastian Pineda
		\end{tabular}
       }

\begin{document}

\maketitle
\date{}
\tableofcontents
\newpage

\begin{abstract}

Abstract goes here

\end{abstract}

\newpage

\section{Introduction}

This thesis was completed in the representation learning lab of Albert-Ludwig-Universität Freiburg.  (Figure~\ref{fig:UniLogo})

\begin{figure}[htb]
  \centering
    \includegraphics[scale=0.35]{images/logo}
    \caption{Logo: Albert-Ludwig-Universität Freiburg}
    \label{fig:UniLogo}
\end{figure}

\subsection{Problem Overview}
\label{ProblemOverviewlabel}
\subsection{Understanding Hyper Parameter Optimization}
\subsection{Understanding Ranking Losses}

\section{Literature Review}
\subsection{Deep Ensembles}
Uncertainty prediction is the key in the paper. \cite{DeepEnsemblePaper}
\subsection{Gaussian Processes}

\section{Existing Methods}

\section{Datasets}
\subsection{HPO\_B Dataset}

\section{Evaluation and Ablation Study}
\subsection{Testing}
\subsection{Ablation}

\section{Results}

\bibliographystyle{plain}
\bibliography{references}

\section{Appendix}
\subsection{More information}

\end{document}
